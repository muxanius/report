
Wireless Sensor Network (WSN) is a set of tiny distributed wireless nodes that
have to collaborate and cooperate on a common distributed application to perform
tasks specified by a user.  Recently, the development of WSNs has received an
increasing interest due to the fast technological advances in the fields of
sensor technology, low power microelectronics, and low energy wireless
communications.

These networks are currently used in wide range of applications in the
scientific, medical, commercial, and military domains, like for instance home
automation, environment monitoring, industrial control, surveillance, security,
healthcare, etc.  In this context, WSNs are intimately tied to and inherently
dependent on an environment they operate in. It leads to a necessity to adapt to
an unpredicted environmental dynamics. Adaptation at run-time provides more
flexibility in such software behavior as energy management, network protocols,
architecture reconfiguration and error handling.  In this regard, WSNs
increasingly need a self-organization, self-configuration and self-adaptation to
changing conditions to ease management and operation.  In this respect, fits the
present work. The remainder of this report is organized as follows. In section
\ref{sec:sa}, we present some related works in a sort of a state of the art
about self configuration of WSNs. This section is subdivided in three parts
describing respectively the concept of self configuration related to the WSNs’
architecture, the applications and services in these networks, and the systems
and networking in these lasts. Overall evaluations about the works are
discussed in Section \ref{sec:ev}.
