%In this work we presented a current state of the art in a field of
%self-configurable Wireless Sensor Network. 

Despite there are a lot of work in the area of self-configurable Wireless Sensor
Networks, the solutions are problem-specific, and there is no holistic approach
for developing self-adaptive WSNs. 

In this report, we classified the state of the art approaches according to three
perspectives. Although, works are not strictly bound to a single perspective,
like shown in \cite{ascent}.

The development of such systems can be a long and expensive process, since the
developer should take care of all possible environmental conditions the system can
experience.  WSNs are usually built on very resource-restricted platforms,
which makes it necessary to predefine all possible behaviour variations, since
the adaptation to unpredicted situations is very expensive in terms of
energy and memory consumption.

Self-adaptation is also expensive in terms of performance, since the system
becomes less responsive during the analysis of the environment. Indeed, there
also could be transitional states, where in the environmental conditions are
changed and the old behaviour is no more suitable, but the new behaviour is not
calculated or applied yet.

Thus, we can consider as
costs of self-adaptivity for WSNs the reduction of both service time and
responsiveness of the system, since additional calculations consume more energy
and time. Therefore, self-adaptation for WSNs becomes an hard choice for such
systems where dynamic changes do not frequently occur. 

On the other hand, self-adaptation schemes lead to relevant advantages in terms
of performance. The most important benefit, targeted by the majority of the
works, is the energy saving. Specifically, a smart usage of resources, such
as switching off some nodes, reducing the transmitting power, and so
on, provides an useful maximization of the battery lifetime by improving the
energy consumption. 

Moreover, the self-adaptation techniques help to perform a more efficient
coverage of the area in the Wireless Sensor Networks. Indeed, the
self-adaptation provides better performance by optimizing the placement of
nodes.  At the same time, this leads to a more reliable system by assuring
fault-tolerance. Interestingly, the nodes which are not involved in the coverage
process, may be activated for supplying the sinks.

Finally, we can state that the most important feature introduced with the
self-adaptive wireless sensor networks is the flexibility, as well as the
ability to proficiently react to the environment changes. In this sense, the
self-adaptation paradigm plays a key-role in terms of modifying system
behaviours to fulfill the network requirements, by monitoring the current
performance of the system.

