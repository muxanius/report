In this work we presented a current state of the art in a field of self-
configurable Wireless Sensor Network. 

Despite there are a lot of work in the area of self-configurable Wireless
Sensor Networks, the solutions are problem-specific, and there is no holistic
approach for developing self-adaptive WSNs. The research has three independent
directions: an architecture adaptation, an application adaptation and a low-
level adaptation. Focusing on the one of the directions, researches absolutely
abandon other two, which makes it difficult to use them all together.

The development of such systems can be a long and expensive process, since
the developer should be care of all possible situations and combinations of
situations, the system can find itself in. WSNs are usually built on very
resource-restricted platforms, which makes it necessary to predefine all
possible behavior variations, since the adaptation to unpredicted situations
could be very expensive in terms of energy and memory consumption.

Self-adaptation is also expensive in terms of performance, since the system
becomes less responsive during the analysis of the environment. There also could
be an intermediate state, when the old behavior is not applicable in the new
situation, but the new behavior is not chosen yet. Thus, the cost of self-
adaptivity for WSNs are decrease of service time and responsiveness of the
system, since additional calculations consume more energy and time.