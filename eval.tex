%In this work we presented a current state of the art in a field of self-configurable Wireless Sensor Network. 

Despite there are a lot of work in the area of self-configurable Wireless Sensor Networks, the solutions are problem-specific, and there is no holistic approach for developing self-adaptive WSNs. As already presented, the research has three independent directions. Focusing on the one of the directions, researches absolutely abandon other two, which makes it difficult to use them all together.

The development of such systems can be a long and expensive process, since the developer should be care of all possible situations and combinations of situations, the system can find itself in. WSNs are usually built on very resource-restricted platforms, which makes it necessary to predefine all possible behaviour variations, since the adaptation to unpredicted situations could be very expensive in terms of energy and memory consumption.

Self-adaptation is also expensive in terms of performance, since the system becomes less responsive during the analysis of the environment. There also could be an intermediate state, when the old behaviour is not applicable in the new situation, but the new behaviour is not chosen yet. Thus, we can consider as costs of self-adaptivity for WSNs the reduction of both service time and responsiveness of the system, since additional calculations consume more energy and time. Therefore, self-adaptation for WSNs becomes an hard choice for such systems where dynamic changes do not frequently occur. 

On the other hand, self-adaptation schemes lead to relevant advantages in terms of performance. The most important benefit, targeted by the majority of the works, is the energy saving. Specifically, a smart using of the resources, such as switching off some of the nodes, reducing the transmitting distances, and so on, as presented in Section \ref{sec:sa}, provides an useful maximization of the battery lifetime by improving the energy consumption. 

Moreover, the self-adaptation techniques help to perform a more efficient coverage of nodes in the standard Wireless Sensor Networks. Indeed, the self-adaptation provides better performance by placing in an optimal way the nodes to cover the system. At the same time, this leads to a more reliable system by assuring the fault-tolerance in the WSNs. Interestingly, the nodes which are not involved in the coverage process, may be activated for supplying the interested nodes.

Finally, we can state that the most important feature introduced with the self-adaptive wireless sensor networks is the flexibility as well as the ability to proficiently react to the environment changes. In this sense, the self-adaptation paradigm plays a key-role in terms of modifying system behaviours by following the  network requirements but, at the same time, monitoring the current performance of the system.
