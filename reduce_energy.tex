\subsubsection*{Use of self-adaptive methodology in wireless sensor networks for
reducing energy consumption}
From an energy saving perspective, interesting approaches have been proposed in
the literature. One of them just exploits the idea that the energy used for
communication in a sensor network can me improved by decreasing the
communication range. The transmission energy is proportional to the distance
between the transmitters. Therefore, in \cite{reduceEnergy} they propose a
self-adaptive methodology which reduce in an optimal fashion the distances
between the main nodes in the network (BS) and the other sensors communicating
with it as well as optimizes the number of active sensors by keeping the
redundancy at good levels.  The authors just divide the approach in two phases:
the former is to identify the right location for the main sensor, which aims at
minimizing the distances with the other sensor nodes, the latter is to switch
off the nodes which are useless for covering the area. Additionally, since the
nodes communicate by broadcasting messages, they suggested a smart solution
which just enables in turn the active node to sense the channel in order to cope
with the collision problem. The results are pretty good in terms of power
saving. Interestingly, the proposed method not only reduces overall energy
consumption of the network but also the lifetime of the nodes is increased
significantly.
