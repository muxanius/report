\subsubsection*{Supporting Lightweight Adaptations in Context-aware
Wireless Sensor Networks}

A common problem with WSN is to deploy an application, or update it, when the
sensor network features a lot of nodes spread over a wide or region. As well,
reprogramming a whole WSN deployed in a inaccessible region, represents a
cumbersome, and often unfeasible, activity. Thus, the challenge is to support
these operations in a feasible way, at the price of the lowest possible
overhead.
\\
Apart from this issues, a typical requirement of applications like
\emph{environmental monitoring} is to adapt the behavior of the nodes to the
dynamics introduced by events, as for instance the presence or people moving
around or not. 
\\
The work of Taherkordi et \emph{al.}\cite{app:taherkordi}, introduces the
\texttt{WiSeKit} distributed middleware, along with the \texttt{ReWiSe} software
component model, as a solution to support a lightweight behavioral adaptation of
sensor networks. The \texttt{WiSeKit} middleware exposes the following services
\begin{itemize}
	\item \emph{Local Reasoning} to update the values of components’ parameters
		based on a local adaptation policy.
	\item \emph{Adaptation Proxy} to receive adaptation request from cluster
		head.
	\item \emph{Component Repository} to temporarily store a new component’s
		image.
	\item \emph{Component Reconfigurator} to load, reload or remove a running
		component.
\end{itemize}
The \texttt{ReWiSe} component model considers the application as an integration
of separate components. A component can be dynamically replaced, Reconfiguration
in  WSN can be a very expensive operation, since it requires to transfer code to
sensor nodes, to save and resume of state information, and eventually restart
the sensor node. The main costs are paid in terms of \emph{energy consumption}
and are proportional to the size of the code images to transfer. For this
reason, the goal of this work has been to develop a software component model
that i) enables the possibility of a fine-grained reconfiguration of the
application, ii) reduce the overhead due to state saving.
The framework has been implemented on top of the \texttt{Contiki} operating
system. Preliminary experiments reported a decrease of energy consumption of
about $75\%$, compared to a common software component model.

