\subsubsection*{COPAL-ML: A Macro Language for Rapid Development of Context-Aware
Applications in Wireless Sensor Networks}

Sehic et. al~\cite{sehic11} propose a Java based API for developing self-adaptive
applications for WSNs. By using this framework developer can specify the way the
contextual data will be collected, processed and used by the application.
COPAL-ML implies such components as:

\begin{itemize}
\item \emph{Context type} specifies the format of the data provided by a WSN.
\item \emph{Publisher} periodically publishes an information about a specific
context type.
\item \emph{Listener} handles an event fired by the publisher and receives a
context type as an argument.
\item \emph{Processor} handles the received data.
\end{itemize}

Along with possible scenario, authors also proposed several processing patterns,
which can be used as solutions for effective self-adaptive application. Despite it
is very promising way towards abstraction of WSN on the high level, the
framework is Java-based. Thus, it is applicable only for sensors with
Java-machine built in. it is also can be used on a computer, but in this case the
communication between the framework and the node remains unclear.

