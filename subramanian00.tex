\subsubsection*{An architecture for building self-configurable systems}

Subramanian et al.~\cite{subramanian00} focused on the self-organizing architecture for WSNs and
proposed components necessary for building such an architecture. The latter
implies following components:

\begin{itemize}
\item \emph{Specialized sensors} for monitoring different physical entities.
\item \emph{Routing sensors} provide a data dissipation and fault tolerance of the network.
\item \emph{Aggregator nodes} combine routing and sensor functionality to
provide a better network flexibility.
\item \emph{Sink nodes} have a high storage capacity, store and process received data.
\end{itemize}
All the components mentioned above are sufficient for building a wide range of
applications, where infrastructure consists of addressing, routing, broadcasting
and multicasting mechanisms.

Given the proposed architectural components, author also propose four steps,
which should be performed by the network to self-organize:

\begin{itemize}
\item \emph{Discovery phase.} Each node discovers its neighbors.
\item \emph{Organizational phase.} Nodes are organizing groups, allocate
addresses, build routing table and construct broadcast tree and graph spanning
all nodes.
\item \emph{Maintenance phase.} Each node keeps track of its energy, constantly
updates routing table, broadcast trees and graphs, sends \emph{I am alive}
message and routing table to its neighbors.
\item \emph{Self-Reorganization phase.} If node detects the failure of its
neighbor, it updates its routing table, or starts \emph{Discovery phase} in case
of the failure of all of the neighbors.
\end{itemize}

The analysis of the approach shows that the hierarchy of the network is strictly
balanced; the complexity of the routing is \emph{O(}log~\emph{n)}; the network is
extremely tolerant to either node or link failures; the uniqueness property is
guaranteed by the presence of a hierarchy; specialized sensors are allowed to be
mobile. There are several weaknesses of the approach, though. Thereby,
the approach is not optimized for extremely dynamic systems, when the network
changes badly or very fast. On the other hand, the required protocol for such
networks is not discussed. Despite the specialized sensors can be mobile, they
can not move beyond the reach of routers. The latter, however are considered static.

The approach is very well fitted for static WSN, such as one for collecting
meteorological data. But it is absolutely not applicable for systems with high
dynamics such as wildlife monitoring~\cite{Pasztor10}, since the network changes
badly and rapidly.
