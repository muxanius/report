\subsubsection*{Proactive Reconfiguration of Wireless Sensor Networks}
\label {ssec:proactive}

Steine at \emph{al.} in \cite{sys:proactive} leverage on the exploitation of a
design-time exploration, to identify a set of operating modes. At run-time the
WSN can adapt its behavior to environmental conditions and QoS fluctuations,
by reconfiguring itself into the most suitable operating mode.  The paper
defines an \emph{operating mode} as a set of values assigned to some
controllable parameters of the network protocols.
The nodes can dynamically change, at run-time, their operating mode according to
specific observable events, that potentially affect the QoS of the WSN. The
events are detected using sensors and/or the current time of the day. This
figures out a \emph{proactive} approach in the self-adaptive behavior.
Concerning the network parameters, it is worth to distinguish between
\emph{local} and \emph{global} parameters. In the former case, the single node
can independently reconfigure itself without affecting the others. For instance,
the tuning of the transmission power. In the latter, changing a global parameter
necessarily requires a synchronization step, in order to maintain the proper
functioning of the WSN. Example of global parameters are the TDMA slot-size and
the sleep-time of the nodes.
Thus, whenever a node detect an event, it can immediately adapt its local
parameters or notify the other nodes about the need of change the current global
mode.
The authors have demonstrated the validity of the approach by testing it in a
"cow-health" and a "office" monitoring scenarios. The evaluation metrics
considered are the average delivery ratio and the average power consumption. In
both the scenarios it has been experienced a significant reduction of power
consumption, comparing the approach to a single configuration (worst-case) not
adaptive design. This, keeping the packets delivery ratio close the values
related to the worst-case based design.

