The presented paper shows some noticeable example of an interesting approach, called self-configuration (or self-adaptation) in the context of Wireless Sensor Networks (WSNs). The survey collects different proposals for any of the above-mentioned categories: the architecture layer, the application layer and the low-level layer.
The work sheds light on what we consider an hot topic in the wireless sensor field, providing for any category the most relevant approaches proposed in the literature.
Finally, advantages and drawback are pointed out in order to evaluate the real impact of introducing the self-adaptive in the common wireless sensors networks, showing a substantially gain in terms of performances for a very dynamic system while an inefficiency when the systems do not face with recurring changes.